\documentclass[a4paper,12pt]{article}
\usepackage[utf8]{inputenc}
\usepackage[russian]{babel}
\usepackage{amsmath}
\usepackage{graphicx}
\usepackage{hyperref}

\title{Основы машинного обучения}
\author{Автор: Иван Иванов}
\date{\today}

\begin{document}

\maketitle

\begin{abstract}
    В этой статье рассматриваются основные понятия и методы машинного обучения. Мы обсудим типы машинного обучения, такие как обучение с учителем и без учителя, а также алгоритмы, которые широко используются для решения задач машинного обучения.
\end{abstract}

\section{Введение}
Машинное обучение (ML) — это область искусственного интеллекта (AI), которая позволяет системам самостоятельно обучаться и делать прогнозы без явного программирования. Используя данные, алгоритмы машинного обучения могут выявлять закономерности и строить модели для принятия решений.

\section{Типы машинного обучения}

Существует несколько типов машинного обучения, каждый из которых используется в зависимости от типа задачи и наличия данных. Наиболее популярные из них:

\subsection{Обучение с учителем}
Обучение с учителем включает использование меток данных. В этом случае у нас есть набор данных, где каждому примеру сопоставлена правильная метка или ответ. Основная цель — построить модель, которая способна предсказывать метку для новых данных. Примеры алгоритмов:
\begin{itemize}
    \item Линейная регрессия
    \item Логистическая регрессия
    \item Деревья решений
    \item Метод опорных векторов (SVM)
\end{itemize}

\subsection{Обучение без учителя}
При обучении без учителя модель работает с немеченными данными и пытается выявить скрытые структуры или закономерности. Примеры задач:
\begin{itemize}
    \item Кластеризация (например, алгоритм K-средних)
    \item Снижение размерности (например, метод главных компонент — PCA)
\end{itemize}

\subsection{Обучение с подкреплением}
В обучении с подкреплением агент взаимодействует с окружающей средой и получает награды или штрафы за свои действия. Цель состоит в максимизации суммарной награды. Этот тип обучения особенно эффективен в задачах, связанных с оптимизацией и управлением.

\section{Основные алгоритмы машинного обучения}

\subsection{Линейная регрессия}
Линейная регрессия — один из самых простых и часто используемых методов машинного обучения. Он предполагает линейную зависимость между входными переменными и целевой переменной. Формула модели:
\[
y = \beta_0 + \beta_1 x_1 + \beta_2 x_2 + \ldots + \beta_n x_n + \epsilon
\]
где \(y\) — целевая переменная, \(x_i\) — входные переменные, \(\beta_i\) — коэффициенты модели, а \(\epsilon\) — ошибка.

\subsection{Деревья решений}
Деревья решений — это структура, которая используется для принятия решений на основе набора правил. Они строятся по принципу разделения данных на подгруппы на основе значений входных переменных.

\begin{figure}[h]
    \centering
    \includegraphics[width=0.5\textwidth]{decision_tree_example.png}
    \caption{Пример дерева решений.}
\end{figure}

\section{Преимущества и недостатки машинного обучения}
Машинное обучение имеет свои сильные и слабые стороны:

\subsection{Преимущества}
\begin{itemize}
    \item Способность обрабатывать большие объемы данных.
    \item Возможность автоматизировать задачи, которые сложно формализовать.
    \item Постоянное улучшение моделей по мере накопления данных.
\end{itemize}

\subsection{Недостатки}
\begin{itemize}
    \item Требует больших наборов данных для качественного обучения.
    \item Модели могут быть сложно интерпретируемыми.
    \item Возможна переобучаемость (overfitting) на обучающем наборе данных.
\end{itemize}

\section{Заключение}
Машинное обучение — это мощный инструмент, который используется во многих областях, таких как медицина, финансы, маркетинг и многое другое. Изучение основных алгоритмов и методов машинного обучения является важным шагом для любого, кто хочет углубиться в эту область.

\begin{thebibliography}{9}
    \bibitem{mitchell} Том Митчелл, \textit{Машинное обучение}, 1997.
    \bibitem{bishop} Кристофер Бишоп, \textit{Распознавание образов и машинное обучение}, 2006.
\end{thebibliography}

\end{document}
